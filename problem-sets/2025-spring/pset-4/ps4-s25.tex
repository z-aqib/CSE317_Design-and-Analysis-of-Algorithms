\documentclass[11pt]{article}
\usepackage{hyperref}
\usepackage{latexsym}
\usepackage{amsmath}
\usepackage{amssymb}
\usepackage{amsthm}
\usepackage{epsfig}
\usepackage{amsfonts}

\usepackage{geometry}
\usepackage{enumitem}

%\geometry{margin=1in}

\newcommand{\handout}[5]{
  \noindent
  \begin{center}
  \framebox{
    \vbox{
      \hbox to 5.78in { {\bf CSE 317 Design and Analysis of Algorithms } \hfill #2 }
      \vspace{4mm}
      \hbox to 5.78in { {\Large \hfill #5  \hfill} }
      \vspace{2mm}
      \hbox to 5.78in { {\em #3 \hfill #4} }
    }
  }
  \end{center}
  \vspace*{4mm}
}

\newcommand{\lecture}[4]{\handout{#1}{#2}{#3}{#4}{#1}}

\newtheorem{theorem}{Theorem}
\newtheorem{corollary}[theorem]{Corollary}
\newtheorem{lemma}[theorem]{Lemma}
\newtheorem{observation}[theorem]{Observation}
\newtheorem{proposition}[theorem]{Proposition}
\newtheorem{definition}[theorem]{Definition}
\newtheorem{claim}[theorem]{Claim}
\newtheorem{fact}[theorem]{Fact}
\newtheorem{assumption}[theorem]{Assumption}

\topmargin 0pt
\advance \topmargin by -\headheight
\advance \topmargin by -\headsep
\textheight 8.9in
\oddsidemargin 0pt
\evensidemargin \oddsidemargin
\marginparwidth 0.5in
\textwidth 6.5in

\parindent 0in
\parskip 1.5ex
%\renewcommand{\baselinestretch}{1.25}

\begin{document}

\lecture{Problem Set 4}{\textit{Assigned: Thursday, 24 Apr}}{Spring 2025}{Due: \textit{11:59 pm Saturday,  May 10}}

\centerline{{\Large To facilitate grading and timely feedback}}
\centerline{{\Large please note that all submissions are through \href{http://gradescope.com}{Gradescope}.}}
\centerline{}
\centerline{{\Large Solve each problem on a new page and put your name on each page.}}
\centerline{{\Large Clearly identify all collaborations and resources used in preparing the solutions.}}
\centerline{{\Large You should typeset your solutions using \LaTeX and correctly link}}
\centerline{{\Large solutions on Gradescope.}}
\begin{enumerate}

\item \textbf{The Final Showdown.} On your quest to prove yourself the \emph{King of the Math Tournament}, you have made it to the final round of battle against the cunning \textbf{Sheikh Chilly}.

Both of you will be given a sheet of paper with a list of $n$ integers $l[1], l[2], \dots, l[n]$ and $m$ pairs of indices $p_i = (i_1, i_2)$ such that $1 \leq i_1 < i_2 \leq n$. Furthermore, a little birdie has told you that the sum of the two indices $i_1$ and $i_2$ satisfy the property
\begin{equation*}
(i_1 + i_2) \,\&\, 1 = 1.
\end{equation*}

In one operation, you can pick any one of the $m$ pairs $p_i = (i_1, i_2)$, and divide both $l[i_1]$ and $l[i_2]$ by any integer $u$, such that $u > 1$ and divides both the numbers,~i.e.,
\begin{equation*}
l[i_k] = \frac{l[i_k]}{u},
\end{equation*}
where $l[i_k]$ is either $l[i_1]$ or $l[i_2]$.

You need to find the \emph{maximum} possible divisions that can be performed.

\begin{enumerate}

\item \emph{(8 points)} Construct a reduction that allows you to solve this problem efficiently using one of the algorithms we have encountered in class. 

\item \emph{(7 points)} Finally, claim your title by writing code for your solution and submit it at:

\begin{center}
\href{https://www.hackerrank.com/ps4-iba}{\texttt{https://www.hackerrank.com/ps4-iba}}
\end{center}

Your solution must pass \textbf{all test cases} for full credit. \textbf{Please ensure that your code submission includes your gradescope username!}

\textbf{Input:}
\begin{itemize}
    \item The first line contains two integers $n$ and $m$ $(2 \leq n \leq 100, 1 \leq m \leq 250)$ - the size of the list and the number of valid index pairs.
    \item The second line contains $n$ integers $l[1], l[2], \dots, l[n]$ $(1 \leq l[i] \leq 10^9)$.
    \item The next $m$ lines each contain two integers $i_k$ and $j_k$ $(1 \leq i_k < j_k \leq n)$, such that $(i_k + j_k) \,\&\, 1 = 1$. All pairs are distinct.
\end{itemize}

\textbf{Output:}

Output a single integer - the maximum number of operations that can be performed.
\newpage
\textbf{Constraints:}

\begin{itemize}
    \item $2 \leq n \leq 100$
    \item $1 \leq m \leq 250$
    \item $1 \leq l[i] \leq 10^9$
    \item $1 \leq i_k < j_k \leq n$
    \item $(i_k + j_k) \,\&\, 1 = 1$
    \item All pairs are distinct
\end{itemize}
\end{enumerate}
\emph{Hint from TA lead Maaz on the question:} What does $(i_1 + i_2) \,\&\, 1 = 1$ imply ?


\item \textbf{There You Go, Gressing Again.} Suppose you have a set of $m$ data points
\begin{equation*}
(x^{(1)}, y^{(1)}), (x^{(2)}, y^{(2)}), \ldots , (x^{(m)}, y^{(m)}).
\end{equation*}
Each data point $(x^{(i)}, y^{(i)})$ has $x^{(i)} \in \mathbb{R}$ and $y^{(i)} \in \mathbb{R}$. Think of $x^{(i)}$ as the \emph{location} where you performed an observation and $y^{(i)}$ as the actual observation. The (vertical) distance of some data point $(x^{(i)}, y^{(i)})$ to the line $y = ax + c$ is
\begin{equation*}
\left| (ax^{(i)} + c) - y^{(i)} \right|.
\end{equation*}
(The distance function varies with the application; often people use ``least-squares
regression'' $((ax^{(i)} + c) - y^{(i)})^2$, but in this problem we consider ``$\textrm{L}_1$-regression''.)

Show how to solve the following problems using linear programming. In each case give
a clear argument why your solution is correct. (The size of a constraint in the LP is
the number of variables in it, and the size of the entire LP is the sum of the sizes of
its constraints.) All distances below refer to vertical distances.

\begin{enumerate}
\item Find a line $y = ax + c$ such that the maximum distance of any data point to the
line is minimized. Your LP should have size $O(m)$.

\item Find a line $y = ax+c$ such that the sum of distances of the m data points to the
line is minimized. Your LP should have size $O(m)$.

\item Now suppose we want to fit a degree-$k$ curve (instead of a line) to the given data.
A degree-$k$ curve is $y = c_k x^k+c_{k-1} x^{k-1}+ \cdots +c_2 x^2+c_1 x+c_0$, where the coefficients $c_i \in \mathbb{R}$. So a degree-$1$ curve is just a line. For a degree-$k$ curve $y = P(x)$, the distance of $(x^{(i)}, y^{(i)})$ to the curve is $\left| P(x^{(i)}) - y^{(i)} \right|$.
Find a degree-$k$ curve minimizing the sum of distances from the $m$ data points to
the curve. Your LP should have size $O(mk)$.

\item Now imagine the data points lie in a higher dimensional space,~e.g., say the data
points actually are $(\mathbf{x^{(i)}}, y^{(i)})$ where now $\mathbf{x^{(i)}} \in \mathbb{R}^n$ and $y^{(i)} \in  \mathbb{R}$. We now want a
degree-$k$ $n$-dimensional curve of the form $y = Q(x_1, x_2, \ldots , x_n)$. For example, a degree-$3$ curve may look like $y = 0.35x^3_1 + 3x^2_1 x_3 - x_2 x_{19} - 0.75x_{53} - 3x_{251} - 5$ or $y = x_1x^2_2 + 19x_7$ or $y = 7x_1x_4x_5+3.2x_2x_5$. The distance of data point $(\mathbf{x^{(i)}}, y^{(i)})$ to
the curve is again $\left| Q(\mathbf{x^{(i)}}) - y^{(i)} \right|$. 
Give an LP of size at most $O(m (n+k)^k)$ size that finds an $n$-dimensional degree-$k$ curve that minimizes the sum of distances of data points from the curve.
\end{enumerate}

\item \textbf{Some Dude's Theorem.} Prove that any point in the convex hull of a set $S \subseteq \mathbb{R}^n$ can be expressed as a convex combination of a set of at most $(n+1)$ points from $S$.

\item \textbf{Some Sums.} The ``3-Sum'' problem is the following: you are given 3 sets $A, B$
and $C$ of $n$ integers each. Your goal is to determine if there exists $a \in A$, $b \in B$ and
$c \in C$ such that $a + b = c$. A simple $O(n^2)$ time algorithm to solve this problem is to hash every element in $C$, and then for each pair of elements $a \in A$ and $b \in B$ to add them up and do a hash lookup on the sum.

It turns out that no $o(n^2)$-time algorithm is known for this problem. However, if the
numbers are all integers between 0 and $m$, then you can solve this problem in time
$O(m \log m)$. How?

\end{enumerate}
\end{document}
